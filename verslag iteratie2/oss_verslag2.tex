\documentclass[i1]{oss}
\setlength{\topmargin}{-.5in}
\setlength{\textheight}{9in}
\setlength{\oddsidemargin}{.100in}
\setlength{\textwidth}{6.25in}
\usepackage [dutch] {babel}
\usepackage{graphicx}
\usepackage{amsmath}
\usepackage{fullpage}
\usepackage{color}
\usepackage{soul}
\usepackage{gensymb}
\usepackage{caption}
\usepackage{subcaption}
\usepackage[section]{placeins}

\begin{document}

\members{Joren Verspeurt {\small \texttt{(r0258417)} } \\
         Sophie Marien {\small \texttt{(s0216517)}}\\
         Stef Noten {\small \texttt{(s0211264)}}\\
         Toon Nolten {\small \texttt{(r0258654)}} \\
         Begeleider: Mario H. C. T.} % teamleden

\maketitlepage
\newpage
\tableofcontents
\pagebreak




%----------------------------------------------------------------------------------------
%	INLEIDING
%----------------------------------------------------------------------------------------
\section*{Inleiding}
\label{ssec:Inleiding}
%introductie en de belangrijkste elementen van ons ontwerp

Het \emph{JUnit} systeem wordt uitgebreid om een automatische test deamon te maken. Deze uitbreiding wordt uitgevoerd op de achtergrond en moet code van een bepaald project in de gaten houden. Wanneer er code gewijzigt wordt in het project dan zal deamon tests uitvoeren en de ontwikkelaar van deze testen op de hoogte brengen.\\


Voor het ontwerp streven we naar lage coupling. Deamon weet alleen maar af van de actieve policies en de policies waarvoor hem op dat moment de statistieken gaat berekenen. Daemon weet wat hij moet weten en niet te veel. Andere belangrijke klassen zijn \textit{Policy}, \textit{Statistic}, \textit{DataCollector}, \textit{AbstractTestCharacteristics}.


%----------------------------------------------------------------------------------------
%	ONTWERP
%----------------------------------------------------------------------------------------
\section{Ontwerp}
\label{ssec:Ontwerp}
%Klassendiagram en interactie diagrammen

In deze sectie bespreken we het algemene ontwerp en de ontwerpbeslissingen. In de eerste subsectie wordt het klassendiagram besproken samen met een aantal verschillen met het vorige klassendiagram.
In de tweede sectie worden de interactiediagrammas besproken die 

%-------------KLASSENDIAGRAMMA-------------------------------
\subsection{Klassendiagramma}
\label{ssec:Klassendiagramma}



%figuur oud klassendiagram
\begin{figure}[tbp]
\begin{center}
    \includegraphics[width=0.8\textwidth]{klassendiagramOud}
    \caption{Grafische User Interface}
	\label{fig:gui}
\end{center}
\end{figure}



%figuur nieuw klassendiagram
\begin{figure}[tbp]
\begin{center}
    \includegraphics[width=0.8\textwidth]{klassendiagram}
    \caption{Grafische User Interface}
	\label{fig:gui}
\end{center}
\end{figure}

%-------------Interactie Diagrammas-------------------------------
\subsection{Interactie diagrammas}
\label{ssec:Interactiedia}


%-------------Ontwerpbeslissingen-------------------------------
\subsection{Ontwerpbeslissingen en patronen}
\label{ssec:Ontwerpbeslissingen}



%----------------------------------------------------------------------------------------
%	TESTEN
%----------------------------------------------------------------------------------------
\section{Testen}
\label{ssec:testen}
%Een hoofdstuk rond testen dat de procedure beschrijft die gevolgd werd bij het testen: wat
%werd getest en hoe?


%----------------------------------------------------------------------------------------
%	PROJECT MANAGMENT
%----------------------------------------------------------------------------------------
\section{Project management}
\label{ssec:Projectmanag}
%taakverdeling elk lid en welke taak
%uren

%TODO tekst

%TODO laatste week aanvullen!!
\begin{table}[h!]
\begin{center}
    \begin{tabular}{ r | c  c  c  c  c  c}
     & Joren & Toon & Stef & Sophie \\ \hline
    Algemeen & 2u00 & 2u00 & 2u00 & 2u00\\
           Tools & 8u30 & 8u30 & 8u00 & 4u30 \\
        Analyse & 10u00 & 10u00 & 12u15 & 12u00 \\
        Ontwerp & 00u00 & 00u00 & 00u00 & 00u00 \\
        Implementeren & 00u00 & 00u00 & 00u00 & 00u00\\
        Verslag & 9u00 & 9u00 & 9u00 & 9u00 \\
        Totaal & 29u30 & 29u30 & 31u15 & 27u30  
    \end{tabular}
    \caption{Overzicht werkuren per onderdeel}
    \label{tab:werkuren}
\end{center}
\end{table}

%TODO figuren updaten
\begin{figure}[h!]
        \centering
        \begin{subfigure}[hb]{0.20\textwidth}
                \centering
                \includegraphics[width=\textwidth]{chart_2}
                \caption{Joren}
        \end{subfigure}%
        \begin{subfigure}[hb]{0.20\textwidth}
                \centering
                \includegraphics[width=\textwidth]{chart_3}
                \caption{Toon}
        \end{subfigure}%
        \begin{subfigure}[hb]{0.20\textwidth}
                \centering
                \includegraphics[width=\textwidth]{chart_4}
                \caption{Stef}
        \end{subfigure}%
        \begin{subfigure}[hb]{0.20\textwidth}
                \centering
                \includegraphics[width=\textwidth]{chart_5}
                \caption{Sophie}
        \end{subfigure}%
                \begin{subfigure}[hb]{0.20\textwidth}
                \centering
                \includegraphics[width=\textwidth]{legende}
                \caption{Legende}
        \end{subfigure}%


 \caption{Weergave van de werkverdeling}
\label{fig:werkverdeling}
\end{figure}





%----------------------------------------------------------------------------------------
%	 CONCLUSIE en DISCUSSIE
%----------------------------------------------------------------------------------------
\section{Conclusie}
\label{ssec:Conclusie}
% Een hoofdstuk met een discussie en conclusie waarin interessante ervaringen, problemen en
%andere opmerkingen omtrent het project worden beschreven



%----------------------------------------------------------------------------------------
%	GLOSSARY
%----------------------------------------------------------------------------------------
\section{Glossary}
\label{ssec:glossary}


Daemon met een policy en een testinformation Klasse en een Statistic information 

Pattern Model-View-Controller
View:  GUI, CLI, passive view
Controller:
Input besturingsview, handeld input/output
Model: Daemon, Policy, TestInfomation, DataCollector, Statistic, TestRun
Daemon is bedoeld als interface voor het model

Verantwoordelijkheden
Policy: ordening, filtering
Statistic: berekent de statistieken, legt een strategie vast om data op te vragen.
TestInformation:  Houdt de statistieken bij (alles dat uit de collectors wordt opgevraagd)
DataCollector: Data verzameling, inpluggen in een TestRun, change code bijhouden (apart in een andere Collector)
Daemon:  Starten van TestRuns
TestRun:  runnen van testen

Implementaties:
Policy als Composit/Decorator/Strategy
We weten niet zeker of het een composit is want er werd getwijfeld tussen Composit en Decorator Pattern. 
Statistic en DataCollector zijn een Strategy
TestInfomation
Eerst mapte Description met een lijst van tuples. Dit is veranderd naar TestSummary
Map<Description,OrderedSet<TestSummary>>
	TestSummary	-Map<Collector, Object>
			- TestID


We streven voor low-coupling. 

Verslag: beginnen met de analyse van wat we nodig hebben en daarna uitleggen hoe we dit oplossen en analyseren en bediscussiëren.

 


\end{document}



